% Užduoties pavadinimas (užduotis), 2009 m. etapas
% ^^^^^^^^^^^^^^^^^^^^^ įrašykite
% Naujoms sąlygoms naudokite šį failą kaip pagrindą.
\documentclass{liostyle}

\renewcommand{\taskname}{užduotis} % naudojamas sistemoje
\renewcommand{\longtaskname}{Užduoties pavadinimas} % pavadinimas

% Čia galite apsirašyti apribojimus ir taip išvengti pakartojimo klaidų,
% jei apribojimai keisis. Dideliems sveikiems skaičiams prasminga
% tūkstančius skirti tarpais -- pagerina skaitomumą.
\newcommand{\maxN}{1\ 000\ 000}
\newcommand{\maxA}{1\ 000}

\begin{document}
Išdėstykite istoriją, kuria bus remiamasi pateikiant užduotį. Tokia
istorija ne visuomet naudojama, tačiau vertinga, kadangi pagyvina
sprendimą ir pagerina uždavinio suvokimą.

\Task
Suformuluokite užduoties sąlygą: ką reikalaujama rasti.

\Input
Aprašykite, kas yra pateikiama kaip įvestis. Paaiškinkite panaudodami
sąvokas iš anksčiau papasakotos istorijos, taip pat specifikuokite
formaliai, idant būtų išvengta nesusipratimų. Čia prasminga pateikti
ir įvairius apribojimus, juos vėliau pakartojant lapo apačioje. Kitoje
pastraipoje pateiktas pavyzdys.

Pirmoje įvesties failo eilutėje įrašytas skaičius $n$~%
($n\le\maxN$), nurodantis toliau einančių eilučių skaičių.
Kiekvienoje iš šių $n$ eilučių pateikiamas skaičius $a_i$~($a_i\le\maxA$),
kur $a_i\le a_j$ kiekvienai porai $(a_i, a_j)$, jei tik $i<j$.

\Output
Pateikite reikalavimus išvesčiai~-- kas turi būti randama bei kaip
pateikiama išvesties faile. Pavyzdys pateikiamas kitoje pastraipoje.

Į išvesties failą įrašykite vieną skaičių, kuris būtų lygus
visų $a_i,\enskip 1\le i\le n$ sumai.

\Examples
% Pateikite pavyzdžius; kiekvienai pavyzdinei įvesčiai parodykite
% teisingą išvestį bei motyvuokite, jeigu tai reikalauja komentarų.
\example%
{
7
5 5
3 6
2 4
4 8
3 6
3 5
6 8
}%
{
4
1
2
4
}%
{
Daugiausia galima pakviesti keturis draugus,
pavyzdžiui 1-ą, 2-ą, 4-ą ir 5-ą. Iš
viso vakarėlyje dalyvaus penki asmenys (su
Justu).

1-asis draugas ateis, nes jis nori, kad
vakarėlyje dalyvautų lygiai 5 asmenys. 

2-asis draugas ateis, nes jis nori, kad
vakarėlyje dalyvautų nuo 3 iki 6 asmenų. 

4-asis ateis, nes jis nori, kad dalyvautų nuo
4 iki 8 asmenų;

5-asis ateis, nes jis nori, kad vakarėlyje
dalyvautų nuo 3 iki 6 asmenų.

Galimi ir kiti sprendiniai.
}

\Grading
% Information about grading (if any).
Už testus, kuriuose $n \le 10\ 000$, galima surinkti iki 50\% balų.

\Constraints
% Reiterate the constraints on the input.
$1\le n \le \maxN$,\enskip
$a_i\le\maxA$.

\end{document}

