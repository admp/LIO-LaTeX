\documentclass{liostyle}

\renewcommand{\taskname}{task} % in lowercase letters
\renewcommand{\longtaskname}{Task Name} % the title
\renewcommand{\version}{1.0}

% Put constraints here, to avoid inconsistency later
\newcommand{\maxN}{1\ 000\ 000}
\newcommand{\maxA}{1\ 000}

\begin{document}

Write a short introductionary story here, if there is one. Usually we
try to have one because it makes the problem easier to understand.

\Task
Put a detailed description of the task, i.e. what is requested to be
found or analysed.

\Input
This part should describe the input format, together with constrants. Next
paragraph provides such an example.

The first line of input will contain the number of items to add 
($\mathbf{N}\le\maxN$).
In the following $\mathbf{N}$ lines $a_i$ ($a_i\le\maxA$) are presented; it holds
$a_i\le a_j$ for every $(a_i, a_j)$ where $i<j$.

\Output
Specify what is to be written to the output file, and what format is to
be used. Example is given in the next paragraph.

Output a single line with the sum of $a_i,\enskip 1\le i\le\mathbf{N}$.

\Examples
% Examples here
\begin{tabular}{|p{4cm}|p{3cm}|p{7.2cm}|}
    \hline
    {\bf \TitleInput} & {\bf \TitleOutput} & {\bf Paaiškinimas} \\
    \hline
    {\tt\obeylines
7
5 5
3 6
2 4
4 8
3 6
3 5
6 8} & {\tt\obeylines
4
1
2
4
5} & {
Daugiausia galima pakviesti keturis draugus,
pavyzdžiui 1-ą, 2-ą, 4-ą ir 5-ą. Iš
viso vakarėlyje dalyvaus penki asmenys (su
Justu).

1-asis draugas ateis, nes jis nori, kad
vakarėlyje dalyvautų lygiai 5 asmenys. 

2-asis draugas ateis, nes jis nori, kad
vakarėlyje dalyvautų nuo 3 iki 6 asmenų. 

4-asis ateis, nes jis nori, kad dalyvautų nuo
4 iki 8 asmenų;

5-asis ateis, nes jis nori, kad vakarėlyje
dalyvautų nuo 3 iki 6 asmenų.

Galimi ir kiti sprendiniai.} \\
    \hline
\end{tabular}

\Grading
Už testus, kuriuose $\mathbf{N} \le 10\ 000$, galima surinkti iki 50\% balų.

\Constraints
$1 \le \mathbf{N} \le \maxN$,\enskip
$2\le m_i\le d_i\le \mathbf{N}+1$.

\end{document}

