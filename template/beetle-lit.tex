\documentclass{liostyle}

\renewcommand{\taskname}{beetle} % in lowercase letters
\renewcommand{\longtaskname}{Vabalas} % the title
\renewcommand{\version}{1.0}

%Constraints
\newcommand{\maxN}{300}
\newcommand{\maxM}{1{,}000{,}000}
\newcommand{\maxX}{10{,}000}

\begin{document}

Vabalas tupi ant plonos horizontalios šakos.
\textit{``Štai aš tupiu ant plonos horizontalios šakos,''} -- galvoja jis, --
\textit{``jaučiuosi, lyg tupėčiau ant $x$ ašies!''}
Tai tikrai matematiškas vabalas.

Ant tos pačios šakos yra $n$ rasos lašų, o kiekviename laše -- $m$ vandens vienetų.
Jų koordinatės vabalo atžvilgiu yra sveikieji skaičiai $x_1, x_2, \ldots, x_n$.

Matyti, kad diena bus karšta.
Per vieną laiko vienetą iš kiekvieno lašo išgaruoja vienas vandens vienetas.
O vabalas iš	troškęs.
Jis toks ištroškęs, kad jei pasiektų rasos lašą, išgertų jį akimirksniu (per nulinį laiką).
Per vieną laiko vienetą vabalas gali nuropoti vieną ilgio vienetą.
Bet ar verta ropoti?
Štai kas neduoda vabalui ramybės!

Taigi jūs turite parašyti programą, kuri pagal duotas lašų koordinates apskaičiuotų,
kokį \emph{didžiausią} vandens kiekį gali suspėti išgerti vabalas.

\subsection*{\Input}

Pradiniai duomenys skaitomi iš standartinio įvesties įrenginio.
Pirmoje eilutėje įrašyti du sveikieji skaičiai $n$ ir $m$.
Kitose $n$ eilučių įrašytos lašų koordinatės $x_1, x_2, \ldots, x_n$ (sveikieji skaičiai).

\subsection*{\Output}

Jūsų programa į standartinį išvesties įrenginį turėtų išvesti vienintelį skaičių --
didžiausią vandens kiekį, kurį gali suspėti išgerti vabalas.

\subsection*{\Example}
\begin{tabular}{|p{5cm}|p{5cm}|}
    \hline
    {\bf \Input} & {\bf \Output} \\
    \hline
    {\tt\obeylines
3 15
6
-3
1} & {\tt\obeylines
25 } \\
    \hline
\end{tabular}

\subsection*{\Constraints}
$0 \le n \le \maxN$,\enspace
$1 \le m \le \maxM$,\enspace
$-\maxX \le x_1,x_2,\ldots,x_n \le \maxX,$\enspace
$x_i \ne x_j$ visiems~$i \ne j$.

\end{document}
